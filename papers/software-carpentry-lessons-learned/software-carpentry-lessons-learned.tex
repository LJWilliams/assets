\documentclass{article}
\begin{document}
\title{Software Carpentry: Lessons Learned}
\date{\today}

\author{
  Katy Huff (University of  Wisconsin / khuff@cae.wisc.edu)\\
  Michael Jackson (Edinburgh Parallel Computing Centre / michaelj@epcc.ed.ac.uk)\\
  Justin Kitzes (University of California, Berkeley / jkitzes@berkeley.edu)\\
  Lex Nederbragt (University of Oslo / lex.nederbragt@ibv.uio.no)\\
  Greg Wilson (Software Carpentry / gvwilson@software-carpentry.org)
}

\maketitle

\begin{abstract}
FIXME: what have \emph{we} learned from Software Carpentry?
\end{abstract}

\section{What We Do That's Different}

\begin{itemize}

  \item KH: It's interactive and peer-taught. It's also directed at scientists. I don't know of anything that currently does all three of those things.

  \item MJ: Teaching tools and technologies by getting the attendees to use them, rather than telling the attendees about them. To get them doing things, rather than subjecting them to hour after hour of slides full of concepts, syntax and command-line excerpts, with the occasional practical thrown in like an oasis in an arid desert.

  \item JK: The most useful/interesting thing is providing an overall structural framework (i.e., the big picture) that shows scientists how different types of computing skills/activities fit together to improve their work. There are lots of workshops on data analysis and other specific topics, but none that I'm aware of that focus on describing the house in which you put the different pieces of furniture.

  \item LN: Change the way one thinks about going about on the command-line/while programming, making it much more logical, robust.

\end{itemize}

\section{What Other People Could Adopt}

\begin{itemize}

  \item KH: I like our github system. I also like your teaching.software-carpentry.org lessons/system/pipeline. If I were to warn another group I would suggest that they seek out a target-audience that is as domain-specific as possible.
    \begin{itemize}
      \item GW: But we \emph{don't} go domain-specific---does that mean you'd change what we do?
    \end{itemize}

  \item MJ: Don't promote software development skills, techniques and tools as being essential because software developers use them and so the attendees should too. Rather, understand what motivates the attendees and promote how these skills, techniques and tools address these motivations (e.g. scripting $\rightarrow$ automation $\rightarrow$ less mistakes + free up time for research, or testing $\rightarrow$ spot mistakes in code $\rightarrow$ prevent errors being introduced into your vital data) etc.

  \item JK: It's extremely important to figure out where one sits in the overall pipeline of the development of computing skills for scientists. In other words, who/what classes or workshops come before you (i.e., basic programming skills, computer literacy) and who/what comes after you (i.e., domain specific data, advanced stats). It's important to be explicit about this both when developing content and in workshops/tutorials themselves.

  \item LN: Teach Software Carpentry \emph{before} people need to code for a living.

\end{itemize}

\section{Citations}

\cite{hannay2009}
\cite{prabhu2011}
\cite{wilson1996}
\cite{wilson2006a}
\cite{wilson2006b}
\cite{wilson2009}

\bibliographystyle{plain}
\bibliography{software-carpentry-lessons-learned}

\end{document}
